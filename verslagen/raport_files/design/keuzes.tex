\subsubsection{Programmeertaal}
We zijn begonnen met het PHP framework \href{http://laravel.com/}{\emph{Laravel (5.0)}}.
Aangezien niemand van ons enige ervaring had met webdevelopment of webdesign was het een
enorme hulp om een soort template te hebben waar we op konden voortgaan. Laravel gebruikt
MVC design, wat we dan ook gevolgd hebben. Voor alle grote klasses werd een Controller aangemaakt.
Deze Controller bevat de voornaamste php code. De resultaten die in de Controller behaald worden,
worden dan gebruikt om Views te creeren die de content geformateerd renderen (als html code).

\subsubsection{Database communicatie}
Om met objecten te kunnen werken worden zogenaamde Models aangemaakt. Dit zijn php klassen die
gebruikt worden als container voor de data die via SQL queries worden opgevraagd.
Bovendien worden deze gebruikt om een zekere mate van beveiliging toe te voegen doordat
in deze models gespecifieerd kan worden welke data wel of niet kan aangepast worden. SQL queries
worden verzameld in een enkele file, en worden doorheen het programma gebruikt om te communiceren
met de database.

