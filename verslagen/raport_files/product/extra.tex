\begin{description}
\item[Turtles] Aangezien "Skulpt" de mogelijkheid biedt om op een makkelijke manier "turtle graphics" te ondersteunen,
    zullen we deze uitbreiding alvast ter beschikking stellen.

\item[Reguliere expressies] Het verwachtte antwoord bij oefeningen aanvaard reguliere expressies die gematched worden
    aan de gegenereerde output.
\item[Vertalingen] Buiten de verplichte Engelse en Nederlanse taal, ondersteunen we enkele tientallen talen.
\item[Messages] Via het messaging systeem kunnen gebruikers onderling met elkaar communiceren.
\item[Notifications] laten gebruikers weten wanneer er relevante dingen gebeuren. Dit zijn onder anderen
    \textsl{"x has sent you a friend request", "x has completed your series y", "series x has been updated, check it out!"}, etc.
\item[Grafieken] Een uitbreiding van de grafieken is dat deze individueel geprint of gedownload kunnen worden in
    allerlei formaten.
\item[Social media login] Voorlopig is enkel Facebook ondersteund, maar dit wordt ongetwijfeld nog uitgebreid.
\subsubsection{Geplande functionaliteit, pls go left}
\item[Lessen] Om beginners op weg te helpen zullen we de cursus Python uit de 1e bachelor gebruiken om echte lessen
    aan te bieden.
\item[Python shell] Een optie om rechtstreeks in een python interpreter te werken.
\item[Groepsleden controle] Mogelijkheid voor founders van groepen om leden te weigeren/uit de groep te verwijderen.
\item[Groepschat] Mogelijkheid in groepen met elkaar te kunnen communiceren, dit om bijvoorbeeld vragen te kunnen
    stellen aan meerdere mensen.
\end{description}
