\begin{description}
\item[Turtles] Aangezien "Skulpt" de mogelijkheid biedt om op een makkelijke manier "turtle graphics" te ondersteunen,
    zullen we deze uitbreiding alvast ter beschikking stellen.
\item[Reguliere expressies] Het verwachtte antwoord bij oefeningen aanvaard reguliere expressies die gematched worden
    aan de gegenereerde output. Enige kennis van regexen is dus vereist voor het opstellen van exercises. Om hierbij te
    helpen wordt een link aangeboden naar een externe website die een duidelijk overzicht aanbiedt met de belangrijkste
    principes.
\item[Vertalingen] Buiten de verplichte Engelse en Nederlanse taal, ondersteunen we enkele tientallen talen.
\item[Messages] Via het messaging systeem kunnen gebruikers onderling met elkaar communiceren. Dit kan rechstreeks naar een
    andere gebruiker of via groepschat.
\item[Notifications] Er is een systeem ge\"{i}mplementeerd om gebruikers te laten weten wanneer er relevante dingen gebeuren. Dit zijn onder anderen
    \textsl{"x has sent you a friend request", "x has completed your series y", "series x has been updated, check it out!"}, etc.
\item[Grafieken] Een uitbreiding van de grafieken is dat deze individueel geprint of gedownload kunnen worden in
    allerlei formaten.
\item[Social media login] Zowel facebook als twitter login wordt ondersteund. Als extra is het bovendien ook mogelijk om de pagina rechtstreeks
    te delen of te 'liken' via facebook. Zo wordt het extra eenvoudig om veel gebruikers aan te trekken.
\item[Guides] Hier is het mogelijk om langere teksten met uitleg over programmeerprincipes te doen. Deze dienen als 'digitaal klaslokaal'.
\item[Groepsleden controle] Mogelijkheid voor founders van groepen om leden te weigeren/uit de groep te verwijderen.
\item[Groepschat] Mogelijkheid in groepen met elkaar te kunnen communiceren, dit om bijvoorbeeld vragen te kunnen stellen aan meerdere mensen.
\item[Zoek functie] Om snel iets te vinden is er een zoekfunctie. Deze werkt over de volledige website en rangschikt de resultaten op een overzichtelijke manier.
    Zoek je bijvoorbeeld een oefening over fibonacci, kan je dat makkelijk via de zoekfunctie doen.
\item[Challenges] Heb je een oefening zeer snel kunnen oplossen, kan je een vriend of vriendin uitdagen dit sneller te doen. Wanneer jij de snelste bent stijg je
    in de ranking ten opzichte van je vrienden!
\item[Leaderboard] Biedt een overzicht van alle gebruikers. Hierbij kan je in een oogopslag zien welke gebruiker hoeveel oefeningen correct heeft kunnen oplossen.
\item[HTML questions] Om interessantere questions op te kunnen stellen kan je hier html code invoegen. Dit geeft de mogelijkheid om niet alleen de layout overzichtelijker
    te maken, maar ook om eventuele afbeeldingen e.d. toe te voegen.
\item[Syntax highlighting] Om goed te leren programmeren is een goede syntax highlighting onontbeerlijk. Dit wordt dan ook ondersteund, specifiek voor de gebruikte programmeertaal.
\item[C++] Indien een gebruiker de beginselen van het programmeren voldoende onder de knie heeft en graag aan iets indrukwekkender begint is C++ de ultieme kandidaat.
    Deze gecompileerde taal is de perfecte manier om performante code te schrijven, gebruikmakend van een interessantere syntax dan Python.
\end{description}
