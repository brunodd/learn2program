\subsubsection{Home page}
Zoals zeer veel van de interface staat de home page nog niet helemaal op punt. De bedoeling is om
in een oogopslag te weten hoe de website georganiseerd is. Op de home page moet zeer duidelijk
zijn waarvoor de website dient, waar je je kan inloggen/aanmelden en naar wat je kan navigeren.

\subsubsection{Navigatie}
Om navigatie te vereenvoudigen is er een permanente balk bovenaan geplaatst om snel naar de voornaamste
delen te navigeren. Eenmaal daar kunnen meer specifieke zaken opgevraagd worden op de webpage zelf.
Zaken die niet van toepassing zijn op een gegeven moment worden verborgen. Een niet-ingelogde bezoeker
krijgt bijvoorbeeld geen optie te zien om een oefeningenreeks te maken.

\subsubsection{Vertalingen}
Om op ieder gewenst ogenblik te kunnen wisselen van taal, is een dropdown-menu voorzien die steeds zichtbaar is.
Indien een gebuiker per ongeluk een taal aanduidt die hem compleet vreemd is, kan hij nog steeds op een vlag klikken
om een courantere taal aan te duiden. Dit is relevant indien een gebruiker bijvoorbeeld op 'chinese' klikt ipv 'dutch'
Het spreekt voor zich dat de meeste gebruikers die 'dutch' willen aanduiden, niets kunnen met chinese tekens.
