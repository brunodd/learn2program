\subsubsection{Homepage}
De homepage is strak maar toch speels ontworpen. Met duidelijke, korte titels trekken we de aandacht van de bezoeker.
Langere, geanimeerde teksten zorgen ervoor dat de gebruiker langer blijft kijken. Op de homepage lichten we kort het concept
van de website toe. De homepage is m.a.w. volledig ontworpen om bezoekers te trekken en bij te houden. Het grafische ontwerp
leent er bovendien toe om extra componenten toe te voegen. Indien er in de toekomst delen toegevoegd moeten worden, kan dit
zonder het grafische concept te verstoren. Een gestreamlinede gebruikservaring primeert!
De tweede belangrijke taak van de homepage is om een kort iest te vertellen over de website. Hier wordt vermeld wat men zoal
kan doen en waarom onze website een meerwaarde vormt voor mensen die graag willen bijleren.

\subsubsection{Navigatie}
Om navigatie te vereenvoudigen is er een navigatiebalk bovenaan geplaatst om snel naar de voornaamste
delen te navigeren. Eenmaal daar kunnen meer specifieke zaken opgevraagd worden via dropdown menu's.
Zaken die niet van toepassing zijn op een gegeven moment worden verborgen. Een niet-ingelogde bezoeker
krijgt bijvoorbeeld geen optie te zien om een oefeningenreeks te maken. Doorheen heel de website is
ervoor gezorgd dat een gebruiker niet moet zoeken. De website moet heel intuitief werken.

\subsubsection{Vertalingen}
Om op ieder gewenst ogenblik te kunnen wisselen van taal, is een dropdown-menu voorzien dat steeds zichtbaar is.
Indien een gebuiker per ongeluk een taal aanduidt die hem compleet vreemd is, kan hij nog steeds op een vlag klikken
om een courantere taal aan te duiden. Dit is relevant indien een gebruiker bijvoorbeeld op 'chinese' klikt ipv 'dutch'
Het spreekt voor zich dat de meeste gebruikers die 'dutch' willen aanduiden, niets kunnen met chinese tekens.
